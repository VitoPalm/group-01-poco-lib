
    \subsection{Iscrizione con Gestione Errori}
        Il bibliotecario avvia la procedura di \textbf{\hyperref[UC:06]{Inserimento Utente}}. Tenta di registrare uno studente inserendo una matricola già esistente o errata. Il sistema, rispettando il requisito \textbf{\hyperref[NRR:02]{Validità Matricola}}, blocca l'operazione e mostra un \textbf{\hyperref[NUR:02]{Errore Esplicativo}} tramite GUI, indicando esattamente cosa correggere. Il bibliotecario corregge i dati e conferma. Il sistema accetta l'inserimento e, prima di sovrascrivere con il dato definitivo, esegue la \textbf{\hyperref[NRR:04]{Creazione Backup}} dei file attuali, poi salva il nuovo utente. Subito dopo, il bibliotecario verifica l'avvenuta operazione effettuando una \textbf{\hyperref[UC:10]{Ricerca Utente}}.

    \subsection{Manutenzione Catalogo}
        Arrivano nuovi libri. Il bibliotecario procede all'\textbf{\hyperref[UC:01]{Inserimento Libro}}. Inserisce un codice ISBN: il sistema controlla la \textbf{\hyperref[NRR:03]{Validità ISBN}} (formato e unicità). Se valido, il libro viene aggiunto e il sistema esegue \textbf{\hyperref[NRR:04]{Creazione Backup}} e \textbf{\hyperref[FDR:05]{salvataggio}}. Successivamente, il bibliotecario identifica un libro obsoleto tramite \textbf{\hyperref[UC:04]{Visita Ordinata}} (per Anno). Seleziona il libro e preme ``Elimina''. In base al requisito \textbf{\hyperref[NUR:03]{Conferma Eliminazione}}, il sistema mostra un pop-up: ``Sei sicuro di voler eliminare questo libro?''. Il bibliotecario conferma. Il sistema rimuove il libro, aggiorna il file di \textbf{\hyperref[NRR:06]{Logging}}, crea un nuovo \textbf{\hyperref[NRR:04]{backup}} di sicurezza e salva lo stato aggiornato dell'archivio.

    \subsection{Gestione Prestiti}
        Un utente chiede un libro. Il bibliotecario usa la \textbf{\hyperref[UC:10]{Ricerca Utente}} e seleziona il profilo. Poi usa la \textbf{\hyperref[UC:05]{Ricerca Libro}} per trovare il volume. Tenta la \textbf{\hyperref[UC:11]{Registrazione Prestito}}. Il sistema rileva che l'utente ha già 3 libri: mostra un \textbf{\hyperref[NUR:02]{Errore Esplicativo}} (es. ``Limite prestiti raggiunto''). L'utente allora restituisce un altro libro che aveva con sé: il bibliotecario esegue \textbf{\hyperref[UC:16]{Registrazione Restituzione}}. Il sistema aggiorna le copie, crea il \textbf{\hyperref[NRR:04]{Backup}} e salva. Ora il bibliotecario riprova la \textbf{\hyperref[UC:11]{Registrazione Prestito}} per il nuovo libro: l'operazione va a buon fine, viene generato un nuovo backup e aggiornato il log.

    \subsection{Crash e Ripristino Dati}
        Il bibliotecario avvia il programma. Durante il \textbf{\hyperref[UC:18]{Caricamento Dati all'Avvio}}, il sistema rileva che il file principale dei dati è corrotto o illegibile. Il sistema mostra un avviso GUI e attiva la procedura di \textbf{\hyperref[NRR:05]{Ripristino da Backup}}. Carica automaticamente l'ultima versione funzionante salvata (come da requisito \textbf{\hyperref[NRR:04]{Creazione Backup}} che mantiene le versioni precedenti). Il bibliotecario accede alla schermata principale e, per sicurezza, verifica tramite il file di \textbf{\hyperref[NRR:06]{Logging delle Azioni}} che le ultime modifiche siano presenti, per confermare che i dati siano stati recuperati correttamente e che il sistema sia operativo.

    \subsection{Modifica Dati Utente}
        Uno studente segnala che la sua email istituzionale è cambiata. Il bibliotecario lo trova tramite \textbf{\hyperref[UC:09]{Visualizzazione Ordinata}} (per Matricola) oppure \textbf{\hyperref[UC:10]{Ricerca Utente}} ed esegue la \textbf{\hyperref[UC:07]{Modifica Utente}}. Il sistema valida la nuova email (\textbf{\hyperref[NRR:01]{Validità Email}}). Se valida, sovrascrive il dato. Immediatamente, il sistema innesca la \textbf{\hyperref[NRR:04]{Creazione Backup}} e il \textbf{\hyperref[FDR:05]{Salvataggio Automatico}}. Per scrupolo, il bibliotecario apre la funzione di visualizzazione del \textbf{\hyperref[NRR:06]{Logging delle Azioni}} per assicurarsi che la modifica sia stata tracciata.

    \subsection{Pulizia Storico Prestiti}
        Il bibliotecario vuole archiviare i vecchi prestiti. Esegue una \textbf{\hyperref[UC:15]{Ricerca Prestito}} (per Data di Restituzione) filtrando quelli passati. Individua un prestito segnato come ``Restituito''. Seleziona l'opzione \textbf{\hyperref[UC:13]{Eliminazione Prestito}}. Il sistema attiva la \textbf{\hyperref[NUR:03]{Conferma Eliminazione}} (``Cancellare definitivamente lo storico di questo prestito?''). Alla conferma positiva, il record viene rimosso. Il sistema esegue il \textbf{\hyperref[FDR:05]{Salvataggio Automatico}} preceduto dalla \textbf{\hyperref[NRR:04]{Creazione Backup}}, garantendo che se si è cancellato per errore, si possa tornare indietro ripristinando il file precedente.

    \subsection{Monitoraggio Ritardi}
        All'apertura, il bibliotecario consulta l'\textbf{\hyperref[UC:14]{Elenco Prestiti Attivi}}. Il sistema applica l' \textbf{\hyperref[FLR:04.1]{Evidenziamento mancate restituzioni}} (es. in rosso) per i prestiti scaduti. Il bibliotecario contatta l'utente in possesso del libro, il quale chiede di posticipare la data di qualche giorno. Il bibliotecario usa la funzione \textbf{\hyperref[UC:12]{Modifica Data di Restituzione Prestito}}. Inserisce una nuova data. Il sistema fa' il \textbf{\hyperref[NRR:04]{Backup}} dei dati attuali e salva la nuova modifica (\textbf{\hyperref[FDR:05]{Salvataggio}}). Tornando all'elenco principale, il prestito non è più evidenziato in rosso, confermando l'aggiornamento.
