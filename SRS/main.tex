\documentclass[12pt,a4paper]{article}
\usepackage[utf8]{inputenc}
\usepackage[T1]{fontenc}
\usepackage[italian]{babel}
\usepackage{geometry}
\usepackage{array}
\usepackage{fancyhdr}
\usepackage{graphicx}
\usepackage{booktabs}
\usepackage{enumitem}
\usepackage{float}
\usepackage[hidelinks]{hyperref}
\usepackage{xcolor}

% Impostazioni pagina
\geometry{left=2.5cm, right=2.5cm, top=3cm, bottom=3cm}
\pagestyle{fancy}
\fancyhf{}
\fancyhead[L]{\textit{Software Requirements Specification}}
\fancyhead[R]{\textit{Gruppo 1}}
\fancyfoot[C]{\thepage}

% Colori personalizzati
\definecolor{sectioncolor}{RGB}{0,102,204}

\title{\textbf{\Large Software Requirements Specification}\\
\large Software per la gestione di una biblioteca}
\author{Autori: Marino Francesco, Pepe Daniele, Orsini Giovanni, Palmieri Vito \\ Gruppo 1}

\begin{document}

    \maketitle
    \thispagestyle{empty}

    \newpage
    \tableofcontents
    \newpage

%===============================================
    \section{Introduzione}\label{sec:introduzione}
%===============================================

    \subsection{Descrizione del progetto}\label{subsec:descrizione-progetto}

%Da inserire

    Il presente documento descrive le specifiche dei requisiti software di un sistema di gestione di una biblioteca (es.\ universitaria).
    Il sistema permette l'inserimento, modifica e cancellazione di libri in una libreria digitale, il tutto memorizzato localmente in un file.
    Permette inoltre di registrare, modificare e rimuovere gli utenti per tracciarne i prestiti.

    \subsection{Convenzioni adottate nel documento}\label{subsec:convenzioni-documento}

    Convenzioni sulla priorità dei requisiti:

    \begin{table}[H]
        \centering

        \begin{tabular}{|c|p{7cm}|c|}
            \hline
            \textbf{Termine} & \textbf{Descrizione} \\
            \hline
            ALTA & Il requisito deve essere necessariamente implementato \\
            \hline
            MEDIO & Il requisito può essere implementato ma non è strettamente necessario\\
            \hline
            BASSA & Il requisito si può implementare se non richiede grandi sforzi\\
            \hline
        \end{tabular}
        \caption{Convenzioni sulla priorità dei requisiti}
        \label{tab:priorities}
    \end{table}

    \noindent Convenzioni sulla numerazione dei requisiti:

    \begin{table}[H]
        \centering
        \begin{tabular}{|c|p{7.7cm}|c|}
            \hline
            \textbf{Numero del requisito} & \textbf{Descrizione} \\
            \hline
            FBR-XX & Functional Book Requirements \\
            \hline
            FUR-XX & Functional User Requirements\\
            \hline
            FLR-XX & Functional Loan Requirements \\
            \hline
            FDR-XX & Functional Data Requirements\\
            \hline
            NUR-XX & Non-Functional Usability Requirements\\
            \hline
            NRR-XX & Non-Functional Reliability Requirements\\
            \hline
            NMR-XX & Non-Functional Maintainability Requirements\\
            \hline
            NPR-XX & Non-Functional Platform Requirements\\
            \hline

        \end{tabular}
        \caption{Convenzioni sulla numerazione dei requisiti}
        \label{tab:req-numbering}
    \end{table}

%===============================================
    \section{Specifica dei requisiti}\label{sec:requirements}
    %===============================================

    \subsection{Requisiti Funzionali (FR)}\label{subsec:functional-req}

    \subsubsection{Gestione dei libri (FBR)}\label{subsubsec:book-management}


    \begin{table}[H]
        \centering
        {\setlength{\extrarowheight}{4pt}
            \begin{tabular}{|l|l|m{10cm}|}
                \hline
                \textbf{ID requisito} & \textbf{Priorità} & \textbf{Descrizione} \\
                \hline
                FBR-01 & ALTA & \textbf{INSERIMENTO LIBRO:} Deve essere possibile inserire un nuovo libro nell'archivio indicando titolo, autori, anno di pubblicazione, codice ISBN e quante copie sono possedute dalla biblioteca \\[20pt]
                \hline
                FBR-02 & ALTA & \textbf{MODIFICA LIBRO:} Deve essere possibile modificare i dati di un libro già presente nell'archivio \\
                \hline
                FBR-03 & ALTA & \textbf{CANCELLAZIONE LIBRO:} Deve essere possibile rimuovere un libro dall'archivio. \\
                \hline
                FBR-04 & ALTA & \textbf{VISITA ORDINATA (PER TITOLO):} Deve essere possibile visualizzare la lista dei libri ordinata alfabeticamente per titolo \\
                \hline
                FBR-04.1 & BASSA & \textbf{VISITA ORDINATA (PER AUTORE, ANNO O NUMERO COPIE):} La lista dei libri può essere visualizzata anche ordinata per autore, anno di pubblicazione o numero di copie disponibili \\
                \hline
                FBR-05 & ALTA & \textbf{RICERCA LIBRO(PER TITOLO, AUTORE, ISBN):} Deve essere possibile cercare un libro nell'archivio inserendo titolo, autore o codice ISBN \\
                \hline
                FBR-05.1 & BASSA & \textbf{RICERCA LIBRO (PER ANNO):} La ricerca può essere fatta anche per anno di pubblicazione \\
                \hline
            \end{tabular}}
        \caption{Requisiti funzionali - Gestione dei libri}
        \label{tab:req-books}
    \end{table}

    \subsubsection{Gestione degli utenti (FUR)}\label{subsubsec:user-management}

    \begin{table}[H]
        \centering
        \begin{tabular}{|l|l|p{9cm}|}
            \hline
            \textbf{ID requisito} & \textbf{Priorità} & \textbf{Descrizione} \\
            \hline
            FUR-01 & ALTA & \textbf{INSERIMENTO UTENTE:} Deve essere possibile registrare un nuovo utente indicando nome, cognome, matricola ed email istituzionale \\
            \hline
            FUR-02 & ALTA & \textbf{MODIFICA UTENTE:} Deve essere possibile modificare i dati di un utente già registrato nell'archivio \\
            \hline
            FUR-03 & ALTA & \textbf{CANCELLAZIONE UTENTE:} Deve essere possibile rimuovere un utente dall'archivio. \\
            \hline
            FUR-04 & ALTA & \textbf{VISITA ORDINATA (ORIDINE LESSICOGRAFICO):} Deve essere possibile visualizzare la lista degli utenti ordinata per cognome e poi per nome \\
            \hline
            FUR-05.1 & BASSA & \textbf{VISITA ORDINATA (PER MATRICOLA):} La lista degli utenti può essere visualizzata anche ordinata per matricola \\
            \hline
            FUR-06 & ALTA & \textbf{RICERCA UTENTE (PER COGNOME O MATRICOLA):} Deve essere possibile cercare un utente nell'archivio inserendo cognome o matricola \\
            \hline
            FUR-06.1 & BASSA & \textbf{RICERCA UTENTE (PER NOME O EMAIL):} La ricerca può essere fatta anche per nome o email istituzionale \\
            \hline
        \end{tabular}
        \caption{Requisiti funzionali - Gestione degli utenti}
        \label{tab:req-users}
    \end{table}

    \subsubsection{Gestione dei prestiti (FLR)}\label{subsubsec:loan-management}

    \begin{table}[H]
        \centering
        \begin{tabular}{|l|l|p{9cm}|}
            \hline
            \textbf{ID requisito} & \textbf{Priorità} & \textbf{Descrizione} \\
            \hline
            FLR-01 & ALTA & \textbf{REGISTRAZIONE PRESTITO:} Deve essere possibile registrare un prestito scegliendo l'utente e il libro, indicando anche quando il libro va restituito.
            Il prestito viene registrato solo se ci sono copie disponibili del libro, e se l'utente non ha già 3 libri in prestito\\
            \hline
            FLR-02 & ALTA & \textbf{ELENCO PRESTITI ATTIVI:} Deve essere possibile visualizzare l'elenco dei prestiti attivi, ordinato per data prevista di restituzione, evidenziando quelli in ritardo \\
            \hline
            FLR-03 & ALTA & \textbf{REGISTRAZIONE RESTITUZIONE:} Deve essere possibile registrare la restituzione di un libro preso in prestito, aggiornando così lo stato del prestito e il numero di copie disponibili del libro \\
            \hline

        \end{tabular}
        \caption{Requisiti funzionali - Gestione dei prestiti}
        \label{tab:req-loans}
    \end{table}


    \subsection{Requisiti sui dati (FDR)}\label{subsec:data-req}

    \begin{table}[H]
        \centering
        \begin{tabular}{|l|l|p{9cm}|}
            \hline
            \textbf{ID requisito} & \textbf{Priorità} & \textbf{Descrizione} \\
            \hline
            FDR-01 & ALTA & \textbf{SALVATAGGIO LIBRI:} Bisogna salvare Titolo, Autori, Anno di pubblicazione, Codice ISBN e numero di copie disponibili di ogni libro\\
            \hline
            FDR-02 & ALTA & \textbf{SALVATAGGIO UTENTI:} Bisogna salvare Nome, Cognome, Matricola, Email istituzionale e Lista dei libri in prestito (con data di restituzione prevista) di ogni utente\\
            \hline
            FDR-03 & ALTA & \textbf{SALVATAGGIO PRESTITI:} Bisogna salvare per ogni prestito l'utente che ha preso il libro, il libro preso in prestito, la data di inizio prestito e la data di restituzione prevista \\
            \hline
            FDR-04 & ALTA & \textbf{CARICAMENTO DATI ALL' AVVIO:} All'avvio del programma, tutti i dati salvati (libri, utenti e prestiti) devono essere caricati automaticamente dall'archivio \\
            \hline
            FDR-05 & MEDIA & \textbf{SALVATAGGIO LIBRI:} Ogni volta che si apporta una modifica (inserimento, modifica, cancellazione di libri, utenti o prestiti), i dati devono essere salvati automaticamente sull'archivio senza richiedere un'azione manuale da parte dell'utente\\
            \hline


        \end{tabular}
        \caption{Requisiti sui dati}
        \label{tab:req-data}
    \end{table}

    \subsection{Requisiti Non Funzionali (NFR)}\label{subsec:nonfunctional-req}

    \subsubsection{Requisiti di Usabilità (NUR)}\label{subsubsec:usability-req}
    \begin{table}[H]
        \centering
        \begin{tabular}{|l|l|p{9cm}|}
            \hline
            \textbf{ID requisito} & \textbf{Priorità} & \textbf{Descrizione} \\
            \hline
            NUR-01 & ALTA & \textbf{INTERFACCIA UTENTE:} Tutte le funzioni (public) del programma devono essere accessibili tramite un'interfaccia grafica (GUI)\\
            \hline
            NUR-02 & MEDIA & \textbf{ERRORI ESPLICATIVI:} In caso di errore (es.\ dati non validi) deve essere mostrato un messaggio chiaro che spiega il problema e come risolverlo\\
            \hline
            NUR-03 & MEDIA & \textbf{CONFERMA ELIMINAZIONE:} Prima di cancellare qualcosa (libro, utente, ecc.) il programma deve sempre chiedere conferma\\
            \hline
        \end{tabular}
        \caption{Requisiti di usabilità}
        \label{tab:req-usability}
    \end{table}

    \subsubsection{Requisiti di Affidabilità (NRR)}\label{subsubsec:reliability-req}
    \begin{table}[H]
        \centering
        \begin{tabular}{|l|l|p{9cm}|}
            \hline
            \textbf{ID requisito} & \textbf{Priorità} & \textbf{Descrizione} \\
            \hline
            NRR-01 & MEDIA & \textbf{VALIDITÀ EMAIL:} Quando si inserisce un nuovo utente, bisogna controllare che l'email istituzionale sia valida e univoca\\
            \hline
            NRR-02 & ALTA & \textbf{VALIDITÀ MATRICOLA:} Quando si inserisce la matricola, bisogna controllare che sia univoca e che segua il formato prestabilito (ad esempio 10 cifre)\\
            \hline
            NRR-03 & MEDIA & \textbf{VALIDITÀ ISBN:} Quando si inserisce un nuovo libro, bisogna controllare che l'ISBN sia univoco e che segua il formato corretto (10 o 13 cifre con eventuali trattini) \\
            \hline
            NRR-04 & MEDIA & \textbf{CREAZIONE BACKUP:} Prima di salvare, il programma crea automaticamente una copia di backup (tenendo le ultime versioni)\\
            \hline
            NRR-05 & MEDIA & \textbf{RIPRISTINO DA BACKUP:} Se ci sono problemi nel leggere il file deve essere possibile ripristinare l'ultima versione funzionante.\\
            \hline
        \end{tabular}
        \caption{Requisiti di affidabilità}
        \label{tab:req-reliability}
    \end{table}

    \subsubsection{Requisiti di Manutenibilità (NMR)}\label{subsubsec:maintainability-req}
    \begin{table}[H]
        \centering
        \begin{tabular}{|l|l|p{9cm}|}
            \hline
            \textbf{ID requisito} & \textbf{Priorità} & \textbf{Descrizione} \\
            \hline
            NMR-01 & MEDIA & \textbf{ORGANIZZAZIONE CODICE:} Il codice va organizzato secondo il pattern MVC (Model-View-Controller) per facilitare modifiche future\\
            \hline
            NMR-02 & ALTA & \textbf{DOCUMENTAZIONE INTERFACCE PUBBLICHE:} Le classi e i metodi pubblici devono essere documentati secondo lo standard di doxygen\\
            \hline
            NMR-02.1 & BASSA & \textbf{DOCUMENTAZIONE INTERFACCE PRIVATE:} Anche le classi e i metodi privati devono essere documentati\\
            \hline
            NMR-03 & MEDIA & \textbf{CONVEZIONE SUI NOMI:} Nomi di variabili e funzioni devono seguire le stesse convenzioni in tutto il progetto (es.\ camelCase per variabili)\\
            \hline
            NMR-04 & ALTA & \textbf{TESTING AUTOMATICO:} Il progetto deve includere test automatici implementati tramite JUnit per le funzionalità principali (non GUI)\\
            \hline
        \end{tabular}
        \caption{Requisiti di manutenibilità}
        \label{tab:req-maintainability}
    \end{table}

    \subsubsection{Requisiti di Piattaforma (NPR)}\label{subsubsec:platform-req}
    \begin{table}[H]
        \centering
        \begin{tabular}{|l|l|p{9cm}|}
            \hline
            \textbf{ID requisito} & \textbf{Priorità} & \textbf{Descrizione} \\
            \hline
            NPR-01 & ALTA & \textbf{LINGUAGGIO:} Il programma deve scritto in Java per garantire portabilità tra diversi sistemi operativi\\
            \hline
            NPR-02 & ALTA & \textbf{BUILD AUTOMATICO:} Deve essere possibile compilare ed eseguire il programma tramite Maven\\
            \hline
        \end{tabular}
        \caption{Requisiti di piattaforma}
        \label{tab:req-platform}
    \end{table}


%===============================================

    \section{Casi d'Uso}\label{sec:use-cases}
    %===============================================

%MODEL
    \subsubsection{UC-XX: Title}
    \begin{itemize}
        \item \textbf{Attori}:
        \item \textbf{Precondizioni}:
        \item \textbf{Postcondizioni}:
        \item \textbf{Scenario Principale}:
        \begin{enumerate}
            \item
            \item
            \item
            \item
            \item
        \end{enumerate}
        \item \textbf{Scenari Alternativi}:
        \begin{itemize}
            \item
        \end{itemize}
    \end{itemize}


    \subsection{Descrizione dei Casi d'Uso Principali}\label{subsec:use-case-descriptions}

    \subsubsection{UC-01: Inserimento Libro}\label{subsubsec:insert-book}
% Diagramma UML + descrizione use case
    \begin{itemize}
        \item \textbf{Attori}: Bibliotecario
        \item \textbf{Precondizioni}: Il bibliotecario ha effettuato l'accesso al pannello di gestione dei libri.
        \item \textbf{Postcondizioni}: Il libro è nell'archivio.
        \item \textbf{Scenario Principale}:
        \begin{enumerate}
            \item Il bibliotecario clicca il bottone per aggiungere un libro.
            \item Il bibliotecario inserisce: titolo, autori, anno, ISBN e numero copie.
            \item Il sistema valida l'ISBN (se implementato: NFR-06).
            \item Il sistema aggiunge il libro all'archivio.
        \end{enumerate}
        \item \textbf{Scenari Alternativi}:
        \begin{itemize}
            \item 3a.\ Il sistema mostra un messaggio di errore se i dati non sono validi e chiede di correggere(se implementato: NFR-02).
        \end{itemize}
    \end{itemize}

    \subsubsection{UC-02: Modifica libro}\label{subsubsec:modify-book}
    \begin{itemize}
        \item \textbf{Attori}: Bibliotecario
        \item \textbf{Precondizioni}: Il libro da modificare esiste nell'archivio.
        \item \textbf{Postcondizioni}: I dati del libro sono aggiornati e salvati.
        \item \textbf{Scenario Principale}:
        \begin{enumerate}
            \item Il bibliotecario seleziona un libro da modificare.
            \item Il bibliotecario clicca il bottone per modificare il libro.
            \item Il bibliotecario aggiorna i campi desiderati.
            \item Il sistema valida i nuovi dati (se implementato: NFR-06).
            \item Il sistema aggiorna le informazioni del libro.
        \end{enumerate}
        \item \textbf{Scenari Alternativi}:
        \begin{itemize}
            \item Il sistema mostra un messaggio di errore se i dati non sono validi e chiede di correggere (se implementato: NFR-02).
        \end{itemize}
    \end{itemize}

    \subsubsection{UC-03: Cancellazione Libro}\label{subsubsec:delete-book}
    \begin{itemize}
        \item \textbf{Attori}: Bibliotecario
        \item \textbf{Precondizioni}: Il libro da eliminare esiste nell'archivio.
        \item \textbf{Postcondizioni}: Il libro non è più presente nell'archivio.
        \item \textbf{Scenario Principale}:
        \begin{enumerate}
            \item Il bibliotecario seleziona il libro da rimuovere.
            \item Il bibliotecario clicca sul pulsante per eliminare il libro.
            \item Il sistema chiede esplicitamente conferma dell'eliminazione (se implementato NFR-03).
            \item Il bibliotecario conferma.
            \item Il sistema rimuove il libro.
        \end{enumerate}
        \item \textbf{Scenari Alternativi}:
        \begin{itemize}
            \item 4a Il bibliotecario nega la conferma e l'operazione viene annullata.
        \end{itemize}
    \end{itemize}

    \subsubsection{UC-04: Visita Ordinata Libri}\label{subsubsec:list-books}
    \begin{itemize}
        \item \textbf{Attori}: Bibliotecario
        \item \textbf{Precondizioni}: L'archivio contiene almeno un libro.
        \item \textbf{Postcondizioni}: Viene mostrata la lista ordinata.
        \item \textbf{Scenario Principale}:
        \begin{enumerate}
            \item Il bibliotecario accede alla sezione libri.
            \item Il sistema mostra la lista ordinata alfabeticamente per Titolo (Default).
        \end{enumerate}
        \item \textbf{Scenari Alternativi}:
        \begin{itemize}
            \item 2a.\ Il bibliotecario sceglie l'ordinamento per Autore, Anno o Numero Copie.
            Il sistema riordina la lista secondo il criterio scelto.
        \end{itemize}
    \end{itemize}


    \subsubsection{UC-05: Ricerca Libro}\label{subsubsec:search-book}
    \begin{itemize}
        \item \textbf{Attori}: Bibliotecario
        \item \textbf{Precondizioni}: Il bibliotecario ha effettuato l'accesso al pannello di gestione dei libri.
        \item \textbf{Postcondizioni}: Vengono mostrati i risultati corrispondenti.
        \item \textbf{Scenario Principale}:
        \begin{enumerate}
            \item Il bibliotecario clicca sulla barra di ricerca.
            \item Il bibliotecario inserisce la stringa di ricerca (Titolo, Autore, Anno o ISBN).
            \item Il sistema filtra l'archivio e mostra i libri che corrispondono ai criteri.
        \end{enumerate}
        \item \textbf{Scenari Alternativi}:
        \begin{itemize}
            \item 3a Se nessun libro corrisponde alla ricerca viene mostrato un messaggio di errore (se implementato: NFR-02).
        \end{itemize}
    \end{itemize}

    \subsubsection{UC-06: Inserimento utente}\label{subsubsec:insert-user}
    \begin{itemize}
        \item \textbf{Attori}: Bibliotecario
        \item \textbf{Precondizioni}: Il bibliotecario ha effettuato l'accesso al pannello gestione utenti.
        \item \textbf{Postcondizioni}: Il nuovo utente è registrato nell'archivio.
        \item \textbf{Scenario Principale}:
        \begin{enumerate}
            \item Il bibliotecario clicca sul tasto per aggiungere un nuovo utente.
            \item Il bibliotecario inserisce Nome, Cognome, Matricola ed Email.
            \item Il sistema valida l'Email (se implementato: NFR-04).
            \item Il sistema valida la Matricola (NFR-05).
            \item Il sistema aggiunge l'utente all'archivio.
        \end{enumerate}
        \item \textbf{Scenari Alternativi}:
        \begin{itemize}
            \item 3a Il sistema mostra un errore se almeno uno dei dati non è valido (se implementato: NFR-02).
        \end{itemize}
    \end{itemize}

    \subsubsection{UC-07: Modifica Utente}\label{subsubsec:modify-user}
    \begin{itemize}
        \item \textbf{Attori}: Bibliotecario
        \item \textbf{Precondizioni}: L'utente da modificare esiste nell'archivio.
        \item \textbf{Postcondizioni}: I dati dell'utente sono aggiornati.
        \item \textbf{Scenario Principale}:
        \begin{enumerate}
            \item Il bibliotecario seleziona un utente.
            \item Il bibliotecario modifica i dati dell'utente.
            \item Il sistema valida i nuovi dati.
            (se implementato: NFR-04) (NFR-05).
            \item Il sistema aggiorna l'archivio.
        \end{enumerate}
        \item \textbf{Scenari Alternativi}:
        \begin{itemize}
            \item 3a Se la modifica crea un conflitto, viene annullata e viene segnalato un errore (se implementato: NFR-02).
        \end{itemize}
    \end{itemize}

    \subsubsection{UC-08: Cancellazione Utente}\label{subsubsec:delete-user}
    \begin{itemize}
        \item \textbf{Attori}: Bibliotecario
        \item \textbf{Precondizioni}: L'utente da eliminare esiste.
        \item \textbf{Postcondizioni}: L'utente non è più nell'archivio.
        \item \textbf{Scenario Principale}:
        \begin{enumerate}
            \item Il bibliotecario seleziona l'utente da rimuovere.
            \item Il bibliotecario clicca sul pulsante per eliminare l'utente.
            \item Il sistema chiede conferma (se implementato NFR-03).
            Il bibliotecario conferma.
            \item Il sistema rimuove l'utente dall'archivio.
        \end{enumerate}
        \item \textbf{Scenari Alternativi}:
        \begin{itemize}
            \item 3a.\ Il sistema impedisce la cancellazione se l'utente ha libri non restituiti.
            \item 4a.\ Il bibliotecario nega la conferma; l'operazione viene annullata.
        \end{itemize}
    \end{itemize}

    \subsubsection{UC-09: Visita Ordinata Utenti}\label{subsubsec:list-users}
    \begin{itemize}
        \item \textbf{Attori}: Bibliotecario
        \item \textbf{Precondizioni}: Il bibliotecario ha effettuato l'accesso alla pagina utenti.
        \item \textbf{Postcondizioni}: La lista è visualizzata secondo il criterio scelto.
        \item \textbf{Scenario Principale}:
        \begin{enumerate}
            \item Il bibliotecario accede alla sezione utenti.
            \item Il sistema mostra gli utenti ordinati per Cognome e poi Nome (Default).
        \end{enumerate}
        \item \textbf{Scenari Alternativi}:
        \begin{itemize}
            \item 2a.\ Il bibliotecario sceglie l'ordinamento per matricola; il sistema aggiorna la lista.
        \end{itemize}
    \end{itemize}

    \subsubsection{UC-10: Ricerca Utente}\label{subsubsec:search-user}
    \begin{itemize}
        \item \textbf{Attori}: Bibliotecario
        \item \textbf{Precondizioni}: Il bibliotecario ha effettuato l'accesso al pannello di gestione utenti.
        \item \textbf{Postcondizioni}: Vengono mostrati i risultati corrispondenti.
        \item \textbf{Scenario Principale}:
        \begin{enumerate}
            \item Il bibliotecario clicca sulla barra di ricerca.
            \item
            \item
            \item
            \item
        \end{enumerate}
        \item \textbf{Scenari Alternativi}:
        \begin{itemize}
            \item
        \end{itemize}
    \end{itemize}

%===============================================

    \section{Tracciabilità (WIP)}\label{sec:traceability}
%===============================================

    \subsection{Tabella di tracciabilità}\label{subsec:trace-table}

    \begin{table}[H]
        \centering
        \begin{tabular}{|c|p{8cm}|c|}
            \hline
            \textbf{ID Requisito} & \textbf{Componente} & \textbf{Casi di test} \\
            \hline
%Da inserire

            \hline
        \end{tabular}
        \caption{Tabella di tracciabilità}
        \label{tab:traceability}
    \end{table}





\end{document}