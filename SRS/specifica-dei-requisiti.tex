%===============================================

\subsection{Requisiti Funzionali (FR)}\label{subsec:functional-req}

\subsubsection{Gestione dei libri (FBR)}\label{subsubsec:book-management}


\begin{table}[H]
    \centering
    \begin{tabularx}{\textwidth}{|l|c|c|X|}
        \hline
        \textbf{ID requisito} & \textbf{Priorità} & \textbf{Rischio} & \textbf{Descrizione} \\
        \hline
        FBR-01\label{FBR:01} & ALTA & BASSO & \textbf{Inserimento Libro:} Deve essere possibile inserire un nuovo libro nell'archivio indicando titolo, autori, anno di pubblicazione, codice ISBN e quante copie sono possedute dalla biblioteca \\
        \hline
        FBR-02\label{FBR:02} & ALTA & BASSO & \textbf{Modifica Libro:} Deve essere possibile modificare i dati di un libro già presente nell'archivio \\
        \hline
        FBR-03\label{FBR:03} & ALTA & BASSO & \textbf{Cancellazione Libro:} Deve essere possibile rimuovere un libro dall'archivio. \\
        \hline
        FBR-04\label{FBR:04} & ALTA & BASSO & \textbf{Visita Ordinata (per Titolo):} Deve essere possibile visualizzare la lista dei libri ordinata alfabeticamente per titolo \\
        \hline
        FBR-04.1\label{FBR:04.1} & BASSA & BASSO & \textbf{Visita Ordinata (per Autore, Anno o Numero Copie):} La lista dei libri può essere visualizzata anche ordinata per autore, anno di pubblicazione o numero di copie disponibili \\
        \hline
        FBR-05\label{FBR:05} & ALTA & ALTO & \textbf{Ricerca Libro (per Titolo, Autore, ISBN):} Deve essere possibile cercare un libro nell'archivio inserendo titolo, autore o codice ISBN \\
        \hline
        FBR-05.1\label{FBR:05.1} & BASSA & ALTO & \textbf{Ricerca Libro (per Anno):} La ricerca può essere fatta anche per anno di pubblicazione \\
        \hline
        \end{tabularx}
    \label{tab:req-books}
\end{table}

\subsubsection{Gestione degli utenti (FUR)}\label{subsubsec:user-management}

    \begin{table}[H]
        \centering
        \begin{tabularx}{\textwidth}{|l|c|c|X|}
            \hline
            \textbf{ID requisito} & \textbf{Priorità} & \textbf{Rischio} & \textbf{Descrizione} \\
            \hline
            FUR-01\label{FUR:01} & ALTA & BASSO & \textbf{Inserimento Utente:} Deve essere possibile registrare un nuovo utente indicando nome, cognome, matricola ed email istituzionale \\
            \hline
            FUR-02\label{FUR:02} & ALTA & BASSO & \textbf{Modifica Utente:} Deve essere possibile modificare i dati di un utente già registrato nell'archivio \\
            \hline
            FUR-03\label{FUR:03} & ALTA & BASSO & \textbf{Cancellazione Utente:} Deve essere possibile rimuovere un utente dall'archivio. \\
            \hline
            FUR-04\label{FUR:04} & ALTA & BASSO & \textbf{Visualizzazione Ordinata (Ordine Lessicografico):} Deve essere possibile visualizzare la lista degli utenti ordinata per cognome e poi per nome \\
            \hline
            FUR-05.1\label{FUR:05.1} & BASSA & BASSO & \textbf{Visualizzazione Ordinata (per Matricola):} La lista degli utenti può essere visualizzata anche ordinata per matricola \\
            \hline
            FUR-06\label{FUR:06} & ALTA & ALTO & \textbf{Ricerca Utente (per Cognome o Matricola):} Deve essere possibile cercare un utente nell'archivio inserendo cognome o matricola \\
            \hline
            FUR-06.1\label{FUR:06.1} & BASSA & ALTO & \textbf{Ricerca Utente (per Nome o Email):} La ricerca può essere fatta anche per nome o email istituzionale \\
            \hline
        \end{tabularx}
        \label{tab:req-users}
    \end{table}

\subsubsection{Gestione dei prestiti (FLR)}\label{subsubsec:loan-management}

    \begin{table}[H]
        \centering
        \begin{tabularx}{\textwidth}{|l|c|c|X|}
            \hline
            \textbf{ID requisito} & \textbf{Priorità} & \textbf{Rischio} & \textbf{Descrizione} \\
            \hline
            FLR-01\label{FLR:01} & ALTA & BASSO & \textbf{Registrazione Prestito:} Deve essere possibile registrare un prestito scegliendo l'utente e il libro, indicando anche quando il libro va restituito.
            Il prestito viene accettato solo se ci sono copie disponibili del libro, e se l'utente ha meno di 3 prestiti attivi\\
            \hline
            FLR-02\label{FLR:02} & ALTA & ALTO & \textbf{Elenco Prestiti Attivi:} Deve essere possibile visualizzare l'elenco dei prestiti attivi, ordinato per data prevista di restituzione, evidenziando quelli in ritardo \\
            \hline
            FLR-03\label{FLR:03} & ALTA &  BASSO &\textbf{Registrazione Restituzione:} Deve essere possibile registrare la restituzione di un libro preso in prestito, aggiornando così lo stato del prestito e il numero di copie disponibili del libro \\
            \hline

    \end{tabularx}
    \label{tab:req-loans}
\end{table}


\subsubsection{Requisiti sui dati (FDR)}\label{subsubsec:data-req}

    \begin{table}[H]
        \centering
        \begin{tabularx}{\textwidth}{|l|c|c|X|}
            \hline
            \textbf{ID requisito} & \textbf{Priorità} & \textbf{Rischio} & \textbf{Descrizione} \\
            \hline
            FDR-01\label{FDR:01} & ALTA & BASSO & \textbf{Salvataggio Libri:} Bisogna salvare Titolo, Autori, Anno di pubblicazione, Codice ISBN e numero di copie disponibili di ogni libro\\
            \hline
            FDR-02\label{FDR:02} & ALTA & BASSO & \textbf{Salvataggio Utenti:} Bisogna salvare Nome, Cognome, Matricola, Email istituzionale e Lista dei libri in prestito (con data di restituzione prevista) di ogni utente\\
            \hline
            FDR-03\label{FDR:03} & ALTA & BASSO & \textbf{Salvataggio Prestiti:} Bisogna salvare per ogni prestito l'utente che ha preso il libro, il libro preso in prestito, la data di inizio prestito e la data di restituzione prevista \\
            \hline
            FDR-04\label{FDR:04} & ALTA & BASSO & \textbf{Caricamento Dati all'Avvio:} All'avvio del programma, tutti i dati salvati (libri, utenti e prestiti) devono essere caricati automaticamente dall'archivio \\
            \hline
            FDR-05\label{FDR:05} & MEDIA & BASSO & \textbf{Salvataggio Automatico:} Ogni volta che si apporta una modifica (inserimento, modifica, cancellazione di libri, utenti o prestiti), i dati devono essere salvati automaticamente sull'archivio senza richiedere un'azione manuale da parte dell'utente\\
            \hline

    \end{tabularx}
    \label{tab:req-data}
\end{table}

\subsection{Requisiti Non Funzionali (NFR)}\label{subsec:nonfunctional-req}

    \subsubsection{Requisiti di Usabilità (NUR)}\label{subsubsec:usability-req}
    \begin{table}[H]
        \centering
        \begin{tabularx}{\textwidth}{|l|c|c|X|}
            \hline
            \textbf{ID requisito} & \textbf{Priorità} & \textbf{Rischio} & \textbf{Descrizione} \\
            \hline
            NUR-01\label{NUR:01} & ALTA & MEDIO & \textbf{Interfaccia Utente:} Tutte le funzioni (public) del programma devono essere accessibili tramite un'interfaccia grafica (GUI)\\
            \hline
            NUR-02\label{NUR:02} & MEDIA & BASSO & \textbf{Errori Esplicativi:} In caso di errore (es.\ dati non validi) deve essere mostrato un messaggio chiaro che spiega il problema e come risolverlo\\
            \hline
            NUR-03\label{NUR:03} & MEDIA & BASSO & \textbf{Conferma Eliminazione:} Prima di cancellare qualcosa (libro, utente, ecc.) il programma deve sempre chiedere conferma\\
            \hline
        \end{tabularx}
        \label{tab:req-usability}
    \end{table}


    \subsubsection{Requisiti di Affidabilità (NRR)}\label{subsubsec:reliability-req}
    \begin{table}[H]
        \centering
        \begin{tabularx}{\textwidth}{|l|c|c|X|}
            \hline
            \textbf{ID requisito} & \textbf{Priorità} & \textbf{Rischio} & \textbf{Descrizione} \\
            \hline
            NRR-01\label{NRR:01} & MEDIA & BASSO & \textbf{Validità Email:} Quando si inserisce un nuovo utente, bisogna controllare che l'email istituzionale sia valida e univoca (non possono esserci più utenti con lo stesso valore)\\
            \hline
            NRR-02\label{NRR:02} & ALTA & BASSO & \textbf{Validità Matricola:} Quando si inserisce la matricola, bisogna controllare che sia univoca (non possono esserci più utenti con lo stesso valore) e che segua il formato prestabilito (ad esempio 10 cifre)\\
            \hline
            NRR-03\label{NRR:03} & MEDIA & BASSO & \textbf{Validità ISBN:} Quando si inserisce un nuovo libro, bisogna controllare che l'ISBN sia univoco e che segua il formato corretto (10 o 13 cifre con eventuali trattini) \\
            \hline
            NRR-04\label{NRR:04} & BASSA & MEDIO & \textbf{Creazione Backup:} Prima di salvare ogni nuova modifica, il programma crea automaticamente una copia di backup (tenendo le ultime versioni)\\
            \hline
            NRR-05\label{NRR:05} & BASSA & MEDIO & \textbf{Ripristino da Backup:} Se ci sono problemi nel leggere il file deve essere possibile ripristinare l'ultima versione funzionante\\
            \hline
            NRR-06\label{NRR:06} & BASSA & MEDIO & \textbf{Logging delle Azioni Effettuate:} Il bibliotecario può verificare le ultime azioni effettuate (aggiunte/modifiche/rimozioni dei dati) in un file dedicato\\
            \hline
        \end{tabularx}
        \label{tab:req-reliability}
    \end{table}

    \subsubsection{Requisiti di Manutenibilità (NMR)}\label{subsubsec:maintainability-req}
    \begin{table}[H]
        \centering
        \begin{tabularx}{\textwidth}{|l|c|c|X|}
            \hline
            \textbf{ID requisito} & \textbf{Priorità} & \textbf{Rischio} & \textbf{Descrizione} \\
            \hline
            NMR-01\label{NMR:01} & MEDIA & BASSO & \textbf{Organizzazione Codice:} Il codice va organizzato secondo il pattern MVC (Model-View-Controller) per facilitare modifiche future\\
            \hline
            NMR-02\label{NMR:02} & ALTA & BASSO & \textbf{Documentazione Interfacce Pubbliche:} Le classi e i metodi pubblici devono essere documentati secondo lo standard di \textit{doxygen}\\
            \hline
            NMR-02.1\label{NMR:02.1} & BASSA & BASSO & \textbf{Documentazione Interfacce Private:} Anche le classi e i metodi privati devono essere documentati\\
            \hline
            NMR-03\label{NMR:03} & MEDIA & BASSO & \textbf{Convenzione sui Nomi:} Nomi di variabili e funzioni devono seguire le stesse convenzioni in tutto il progetto (es.\ camelCase per variabili)\\
            \hline
            NMR-04\label{NMR:04} & ALTA & BASSO & \textbf{Testing Automatico:} Il progetto deve includere test automatici implementati tramite \textit{JUnit} per le funzionalità principali (non GUI)\\
            \hline
        \end{tabularx}
        \label{tab:req-maintainability}
    \end{table}

    \subsubsection{Requisiti di Piattaforma (NPR)}\label{subsubsec:platform-req}
    \begin{table}[H]
        \centering
        \begin{tabularx}{\textwidth}{|l|c|c|X|}
            \hline
            \textbf{ID requisito} & \textbf{Priorità} & \textbf{Rischio} & \textbf{Descrizione} \\
            \hline
            NPR-01\label{NPR:01} & ALTA & BASSO & \textbf{Linguaggio:} Il programma deve scritto in Java per garantire portabilità tra diversi sistemi operativi\\
            \hline
            NPR-02\label{NPR:02} & ALTA & BASSO & \textbf{Build Automatico:} Deve essere possibile compilare ed eseguire il programma tramite Maven\\
            \hline
        \end{tabularx}
        \label{tab:req-platform}
    \end{table}

%===============================================