%===============================================

    \subsection{Descrizione dei Casi d'Uso sulla Gestione dei Libri}\label{subsec:books-use-cases}

    \begin{figure}[H]
        \centering
        \includegraphics[width=1\linewidth]{UML-books@4x}
        \label{fig:books-uml}
    \end{figure}

    \subsubsection{UC-01: Inserimento Libro}\label{UC:01}

    \begin{itemize}
        \item \textbf{Attori}: Bibliotecario
        \item \textbf{Precondizioni}: Il bibliotecario ha effettuato l'accesso al pannello di gestione dei libri.
        \item \textbf{Postcondizioni}: Il libro è nell'archivio.
        \item \textbf{Requisiti Implementati}: \hyperref[FBR:01]{FBR-01}, \hyperref[FDR:01]{FDR-01}, \hyperref[NUR:01]{NUR-01}
        \item \textbf{Scenario Principale}:
        \begin{enumerate}
            \item Il bibliotecario clicca il bottone per aggiungere un libro.
            \item Il bibliotecario inserisce: titolo, autori, anno, ISBN e numero copie: il sistema valida l'ISBN (se implementato: \hyperref[NRR:03]{NRR-03}).
            \item Il bibliotecario clicca il bottone per salvare.
            \item Il sistema aggiunge il libro all'archivio.
        \end{enumerate}
        \item \textbf{Scenari Alternativi}:
        \begin{itemize}
            \item[3a.] Il sistema mostra un messaggio di errore se l'ISBN non è valido (se implementato: \hyperref[NUR:02]{NUR-02}).
        \end{itemize}
    \end{itemize}

    \subsubsection{UC-02: Modifica libro}\label{UC:02}
    \begin{itemize}
        \item \textbf{Attori}: Bibliotecario
        \item \textbf{Precondizioni}: Il libro da modificare esiste nell'archivio.
        \item \textbf{Postcondizioni}: I dati del libro sono aggiornati e salvati.
        \item \textbf{Requisiti Implementati}: \hyperref[FBR:02]{FBR-02}, \hyperref[FDR:01]{FDR-01}, \hyperref[NUR:01]{NUR-01}
        \item \textbf{Scenario Principale}:
        \begin{enumerate}
            \item Il bibliotecario seleziona un libro da modificare.
            \item Il bibliotecario clicca il bottone per visualizzare le proprietà del libro.
            \item Il bibliotecario clicca il pulsante per entrare in modalità di modifica dei campi.
            \item Il bibliotecario aggiorna i campi desiderati: il sistema valida i nuovi dati (se implementato: \hyperref[NRR:03]{NRR-03}).
            \item Il bibliotecario clicca il bottone per il salvataggio dei dati.
            \item Il sistema aggiorna le informazioni del libro.
        \end{enumerate}
        \item \textbf{Scenari Alternativi}:
        \begin{itemize}
            \item[5a.] Il sistema mostra un messaggio di errore se l'ISBN non è valido (se implementato: \hyperref[NUR:02]{NUR-02}).
        \end{itemize}
    \end{itemize}

    \subsubsection{UC-03: Eliminazione Libro}\label{UC:03}
    \begin{itemize}
        \item \textbf{Attori}: Bibliotecario
        \item \textbf{Precondizioni}: Il libro da eliminare esiste nell'archivio e non ci sono prestiti attivi che lo coinvolgono.
        \item \textbf{Postcondizioni}: Il libro non è più presente nell'archivio.
        \item \textbf{Requisiti Implementati}: \hyperref[FBR:03]{FBR-03}, \hyperref[NUR:01]{NUR-01}
        \item \textbf{Scenario Principale}:
        \begin{enumerate}
            \item Il bibliotecario seleziona il libro da rimuovere.
            \item Il bibliotecario clicca il bottone per visualizzare le proprietà del libro.
            \item Il bibliotecario clicca sul pulsante per eliminare il libro (è possibile solo se sono rispettate le precondizioni: vedi \hyperref[UC:03.1a]{scenario 3a}).
            \item Il sistema chiede esplicitamente conferma dell'eliminazione (se implementato \hyperref[NUR:03]{NUR-03}).
            \item Il bibliotecario conferma.
            \item Il sistema rimuove il libro.
        \end{enumerate}
        \item \textbf{Scenari Alternativi}:
        \begin{itemize}
            \item[3a.]\label{UC:03.1a} Il bottone elimina è disabilitato nel caso non siano soddisfatte le precondizioni (on hover se implementato: \hyperref[NUR:02]{NUR-02}).
            \item[4a.] Il bibliotecario nega la conferma e l'operazione viene annullata.
        \end{itemize}
    \end{itemize}

    \subsubsection{UC-04: Visualizzazione Ordinata Libri}\label{UC:04}
    \begin{itemize}
        \item \textbf{Attori}: Bibliotecario
        \item \textbf{Precondizioni}: L'archivio contiene almeno un libro.
        \item \textbf{Postcondizioni}: Viene mostrata la lista ordinata.
        \item \textbf{Requisiti Implementati}: \hyperref[FBR:04]{FBR-04}, \hyperref[FBR:04.1]{FBR-04.1}, \hyperref[NUR:01]{NUR-01}
        \item \textbf{Scenario Principale}:
        \begin{enumerate}
            \item Il bibliotecario accede alla sezione libri.
            \item Il sistema mostra la lista ordinata alfabeticamente per Titolo (Default).
        \end{enumerate}
        \item \textbf{Scenari Alternativi}:
        \begin{enumerate}
            \item[2a.] Il bibliotecario può richiedere l'ordinamento per Autore, Anno, ISBN o Numero di Copie, premendo l'header della colonna corrispondente alla proprietà, e ri-premendo per invertire l'ordinamento rispetto alla stessa proprietà. (se implementato: \hyperref[FBR:04.1]{FBR-04.1})
        \end{enumerate}
    \end{itemize}

    \subsubsection{UC-05: Ricerca Libro}\label{UC:05}
    \begin{itemize}
        \item \textbf{Attori}: Bibliotecario
        \item \textbf{Precondizioni}: Il bibliotecario ha effettuato l'accesso al pannello di gestione dei libri.
        \item \textbf{Postcondizioni}: Vengono mostrati i risultati corrispondenti.
        \item \textbf{Requisiti Implementati}: \hyperref[FBR:05]{FBR-05}, \hyperref[FBR:05.1]{FBR-05.1}, \hyperref[NUR:01]{NUR-01}
        \item \textbf{Scenario Principale}:
        \begin{enumerate}
            \item Il bibliotecario clicca sulla barra di ricerca.
            \item Il bibliotecario inserisce la stringa di ricerca (Titolo, Autore, Anno o ISBN).
            \item Il sistema filtra l'archivio e mostra i libri che corrispondono ai criteri.
        \end{enumerate}
        \item \textbf{Scenari Alternativi}:
        \begin{itemize}
            \item[3a.] Se nessun libro corrisponde alla ricerca viene mostrato un messaggio di errore (se implementato: \hyperref[NUR:02]{NUR-02}).
        \end{itemize}
    \end{itemize}


    \subsection{Descrizione dei Casi d'Uso sulla Gestione degli Utenti}\label{subsec:users-use-cases}

    \begin{figure}[H]
        \centering
        \includegraphics[width=1\linewidth]{UML-users@4x}
        \label{fig:gestione-utenti}
    \end{figure}

    \subsubsection{UC-06: Inserimento utente}\label{UC:06}
    \begin{itemize}
        \item \textbf{Attori}: Bibliotecario
        \item \textbf{Precondizioni}: Il bibliotecario ha effettuato l'accesso al pannello gestione utenti.
        \item \textbf{Postcondizioni}: Il nuovo utente è registrato nell'archivio.
        \item \textbf{Requisiti Implementati}: \hyperref[FUR:01]{FUR-01}, \hyperref[FDR:02]{FDR-02}, \hyperref[NUR:01]{NUR-01}
        \item \textbf{Scenario Principale}:
        \begin{enumerate}
            \item Il bibliotecario clicca sul tasto per aggiungere un nuovo utente.
            \item Il bibliotecario inserisce Nome, Cognome, Matricola ed Email: il sistema valida l'Email (se implementato: \hyperref[NRR:01]{NRR-01}) e la Matricola (\hyperref[NRR:02]{NRR-02}).
            \item Il bibliotecario clicca il pulsante per il salvataggio dei dati.
            \item Il sistema aggiunge l'utente all'archivio.
        \end{enumerate}
        \item \textbf{Scenari Alternativi}:
        \begin{itemize}
            \item[3a.] Il sistema mostra un errore se almeno uno dei dati non è valido (se implementato: \hyperref[NUR:02]{NUR-02}).
        \end{itemize}
    \end{itemize}

    \subsubsection{UC-07: Modifica Utente}\label{UC:07}
    \begin{itemize}
        \item \textbf{Attori}: Bibliotecario
        \item \textbf{Precondizioni}: L'utente da modificare esiste nell'archivio.
        \item \textbf{Postcondizioni}: I dati dell'utente sono aggiornati.
        \item \textbf{Requisiti Implementati}: \hyperref[FUR:02]{FUR-02}, \hyperref[FDR:02]{FDR-02}, \hyperref[NUR:01]{NUR-01}
        \item \textbf{Scenario Principale}:
        \begin{enumerate}
            \item Il bibliotecario seleziona un utente.
            \item Il bibliotecario clicca il bottone per visualizzare le proprietà dell'utente.
            \item Il bibliotecario modifica i dati dell'utente: il sistema valida i nuovi dati (se implementato: \hyperref[NRR:01]{NRR-01}, \hyperref[NRR:02]{NRR-02}).
            \item Il bibliotecario clicca il pulsante per il salvataggio dei dati.
            \item Il sistema aggiorna l'archivio.
        \end{enumerate}
        \item \textbf{Scenari Alternativi}:
        \begin{itemize}
            \item[3a.] Il sistema mostra un errore se almeno uno dei dati non è valido (se implementato: \hyperref[NUR:02]{NUR-02}).
        \end{itemize}
    \end{itemize}

    \subsubsection{UC-08: Eliminazione Utente}\label{UC:08}
    \begin{itemize}
        \item \textbf{Attori}: Bibliotecario
        \item \textbf{Precondizioni}: L'utente da eliminare esiste e non ha libri non restituiti.
        \item \textbf{Postcondizioni}: L'utente non è più nell'archivio.
        \item \textbf{Requisiti Implementati}: \hyperref[FUR:03]{FUR-03}, \hyperref[NUR:01]{NUR-01}
        \item \textbf{Scenario Principale}:
        \begin{enumerate}
            \item Il bibliotecario seleziona l'utente da rimuovere.
            \item Il bibliotecario clicca il bottone per visualizzare le proprietà dell'utente.
            \item Il bibliotecario clicca sul pulsante per eliminare l'utente (è possibile solo se sono rispettate le precondizioni: vedi \hyperref[UC:08.3a]{scenario 3a}).
            \item Il sistema chiede conferma (se implementato: \hyperref[NUR:03]{NUR-03}).
            \item Il bibliotecario conferma.
            \item Il sistema rimuove l'utente dall'archivio.
        \end{enumerate}
        \item \textbf{Scenari Alternativi}:
        \begin{itemize}
            \item[3a.]\label{UC:08.3a} Il bottone elimina è disabilitato nel caso non siano soddisfatte le precondizioni (on hover se implementato: \hyperref[NUR:02]{NUR-02}).
            \item[4a.] Il bibliotecario nega la conferma; l'operazione viene annullata.
        \end{itemize}
    \end{itemize}

    \subsubsection{UC-09: Visualizzazione Ordinata Utenti}\label{UC:09}
    \begin{itemize}
        \item \textbf{Attori}: Bibliotecario
        \item \textbf{Precondizioni}: Il bibliotecario si trova nella scheda utenti.
        \item \textbf{Postcondizioni}: La lista è visualizzata secondo il criterio scelto.
        \item \textbf{Requisiti Implementati}: \hyperref[FUR:04]{FUR-04}, \hyperref[FUR:04.1]{FUR-04.1}, \hyperref[NUR:01]{NUR-01}
        \item \textbf{Scenario Principale}:
        \begin{enumerate}
            \item Il bibliotecario accede alla sezione utenti.
            \item Il sistema mostra gli utenti ordinati per Cognome e poi Nome (Default).
        \end{enumerate}
        \item \textbf{Scenari Alternativi}:
        \begin{enumerate}
            \item[2a.] Il bibliotecario può richiedere l'ordinamento per Matricola, premendo l'header della colonna corrispondente alla proprietà, e ri-premendo per invertire l'ordinamento rispetto alla stessa proprietà. (se implementato: \hyperref[FUR:04.1]{FUR-04.1})
        \end{enumerate}
    \end{itemize}

    \subsubsection{UC-10: Ricerca Utente}\label{UC:10}
    \begin{itemize}
        \item \textbf{Attori}: Bibliotecario
        \item \textbf{Precondizioni}: Il bibliotecario ha effettuato l'accesso al pannello di gestione utenti.
        \item \textbf{Postcondizioni}: Vengono mostrati i risultati corrispondenti.
        \item \textbf{Requisiti Implementati}: \hyperref[FUR:05]{FUR-05}, \hyperref[FUR:05.1]{FUR-05.1}, \hyperref[NUR:01]{NUR-01}
        \item \textbf{Scenario Principale}:
        \begin{enumerate}
            \item Il bibliotecario clicca sulla barra di ricerca.
            \item Il bibliotecario inserisce la stringa di ricerca (Nome, Cognome, Matricola o Email).
            \item Il sistema filtra l'archivio e mostra gli utenti che corrispondono ai criteri.
        \end{enumerate}
        \item \textbf{Scenari Alternativi}:
        \begin{itemize}
            \item[3a.] Se nessun utente corrisponde alla ricerca viene mostrato un messaggio di errore (se implementato: \hyperref[NUR:02]{NUR-02}).
        \end{itemize}
    \end{itemize}


    \subsection{Descrizione dei Casi d'Uso sulla Gestione dei Prestiti}\label{subsec:lendings-use-cases}

    \begin{figure}[H]
        \centering
        \includegraphics[width=1\linewidth]{UML-lendings@4x}
        \label{fig:gestionePrestiti}
    \end{figure}

    \subsubsection{UC-11: Registrazione Prestito}\label{UC:11}
    \begin{itemize}
        \item \textbf{Attori}: Bibliotecario
        \item \textbf{Precondizioni}: Utente e Libro esistono, l'utente ha meno di 3 libri in prestito e ci sono copie disponibili del libro.
        \item \textbf{Postcondizioni}: Prestito registrato, copie libro decrementate, lista prestiti utente aggiornata.
        \item \textbf{Requisiti Implementati}: \hyperref[FLR:01]{FLR-01}, \hyperref[FDR:03]{FDR-03}, \hyperref[NUR:01]{NUR-01}
        \item \textbf{Scenario Principale}:
        \begin{enumerate}
            \item Il bibliotecario seleziona un Utente e un Libro (è possibile la sola selezione di libri e utenti che rispettano le precondizioni - vedere \hyperref[UC:11.1a]{scenario 1a}).
            \item Il bibliotecario indica la data di restituzione prevista.
            \item Il bibliotecario clicca il pulsante per il salvataggio dei dati.
            \item Il sistema registra il prestito (associa il libro all'utente), decrementa le copie del libro, incrementa i prestiti dell'utente.
        \end{enumerate}
        \item \textbf{Scenari Alternativi}:
        \begin{itemize}
            \item[1a.]\label{UC:11.1a} L'utente/libro con precondizioni non soddisfatte non è selezionabile (mostrato on hover se implementato: \hyperref[NUR:02]{NUR-02})
            \item[3a.] Il sistema mostra un errore se la data non è valida (se implementato: \hyperref[NUR:02]{NUR-02}).
        \end{itemize}
    \end{itemize}

    \subsubsection{UC-12: Modifica Data di Restituzione Prestito}\label{UC:12}
    \begin{itemize}
        \item \textbf{Attori}: Bibliotecario
        \item \textbf{Precondizioni}: Il prestito da modificare esiste nell'archivio.
        \item \textbf{Postcondizioni}: I dati del prestito sono aggiornati.
        \item \textbf{Requisiti Implementati}: \hyperref[FLR:02]{FLR-02}, \hyperref[FDR:03]{FDR-03}, \hyperref[NUR:01]{NUR-01}
        \item \textbf{Scenario Principale}:
        \begin{enumerate}
            \item Il bibliotecario seleziona un prestito.
            \item Il bibliotecario clicca il bottone per visualizzare le proprietà del prestito.
            \item Il bibliotecario modifica la data di restituzione prevista.
            \item Il bibliotecario clicca il pulsante per il salvataggio dei dati.
            \item Il sistema aggiorna l'archivio.
        \end{enumerate}
        \item \textbf{Scenari Alternativi}:
        \begin{itemize}
            \item[3a.] Il sistema mostra un errore se la data non è valida (se implementato: \hyperref[NUR:02]{NUR-02}).
        \end{itemize}
    \end{itemize}

    \subsubsection{UC-13: Eliminazione Prestito}\label{UC:13}
    \begin{itemize}
        \item \textbf{Attori}: Bibliotecario
        \item \textbf{Precondizioni}: Il prestito da eliminare esiste nell'archivio ed è stato restituito.
        \item \textbf{Postcondizioni}: Il prestito non è più nell'archivio.
        \item \textbf{Requisiti Implementati}: \hyperref[FLR:03]{FLR-03}, \hyperref[NUR:01]{NUR-01}
        \item \textbf{Scenario Principale}:
        \begin{enumerate}
            \item Il bibliotecario seleziona il prestito da eliminare.
            \item Il bibliotecario clicca il bottone per visualizzare le proprietà del prestito.
            \item Il bibliotecario clicca sul pulsante per eliminare il prestito (è possibile solo se sono rispettate le precondizioni: vedi \hyperref[UC:13.3a]{scenario 3a}).
            \item Il sistema chiede conferma (se implementato: \hyperref[NUR:03]{NUR-03}).
            \item Il bibliotecario conferma.
            \item Il sistema rimuove il prestito dall'archivio.
        \end{enumerate}
        \item \textbf{Scenari Alternativi}:
        \begin{itemize}
            \item[3a.]\label{UC:13.3a} Il bottone elimina è disabilitato nel caso non siano soddisfatte le precondizioni (on hover se implementato: \hyperref[NUR:02]{NUR-02}).
            \item[4a.] Il bibliotecario nega la conferma; l'operazione viene annullata.
        \end{itemize}
    \end{itemize}

    \subsubsection{UC-14: Visualizzazione Prestiti}\label{UC:14}
    \begin{itemize}
        \item \textbf{Attori}: Bibliotecario
        \item \textbf{Precondizioni}: Il bibliotecario si trova nella scheda prestiti.
        \item \textbf{Postcondizioni}: La lista dei prestiti è visualizzata con evidenziazione ritardi.
        \item \textbf{Requisiti Implementati}: \hyperref[FLR:04]{FLR-04}, \hyperref[FLR:04.1]{FLR-04.1}, \hyperref[NUR:01]{NUR-01}
        \item \textbf{Scenario Principale}:
        \begin{enumerate}
            \item Il bibliotecario accede alla sezione Prestiti.
            \item Il sistema mostra l'elenco ordinato per data di restituzione prevista.
            \item Il sistema evidenzia i prestiti la cui data prevista è antecedente alla data odierna.
        \end{enumerate}
        \item \textbf{Scenari Alternativi}:
        \begin{itemize}
            \item[2a.] Non vengono visualizzati prestiti nel caso in cui non ci siano.
        \end{itemize}
    \end{itemize}

    \subsubsection{UC-15: Ricerca Prestito}\label{UC:15}
    \begin{itemize}
        \item \textbf{Attori}: Bibliotecario
        \item \textbf{Precondizioni}: Il bibliotecario ha effettuato l'accesso al pannello di gestione prestiti.
        \item \textbf{Postcondizioni}: Vengono mostrati i risultati corrispondenti.
        \item \textbf{Requisiti Implementati}: \hyperref[FLR:05]{FLR-05}, \hyperref[FLR:05.1]{FLR-05.1}, \hyperref[NUR:01]{NUR-01}
        \item \textbf{Scenario Principale}:
        \begin{enumerate}
            \item Il bibliotecario clicca sulla barra di ricerca.
            \item Il bibliotecario inserisce la stringa di ricerca (Data di restituzione, Titolo/Autori/ISBN
del libro e/o Nome/Cognome/ID utente).
            \item Il sistema filtra l'archivio e mostra i prestiti che corrispondono ai criteri.
        \end{enumerate}
        \item \textbf{Scenari Alternativi}:
        \begin{itemize}
            \item[3a.] Se nessun prestito corrisponde alla ricerca viene mostrato un messaggio di errore (se implementato: \hyperref[NUR:02]{NUR-02}).
        \end{itemize}
    \end{itemize}

    \subsubsection{UC-16: Registrazione Restituzione}\label{UC:16}
    \begin{itemize}
        \item \textbf{Attori}: Bibliotecario
        \item \textbf{Precondizioni}: Esiste un prestito attivo per quell'utente/libro.
        \item \textbf{Postcondizioni}: Prestito eliminato, copie libro incrementate.
        \item \textbf{Requisiti Implementati}: \hyperref[FLR:06]{FLR-06}, \hyperref[FDR:02]{FDR-02}, \hyperref[FDR:03]{FDR-03}, \hyperref[NUR:01]{NUR-01}
        \item \textbf{Scenario Principale}:
        \begin{enumerate}
            \item Il bibliotecario seleziona un prestito attivo dalla sezione prestiti.
            \item Il bibliotecario clicca sul tasto per registrare la restituzione.
            \item Il sistema aggiorna lo stato del prestito.
            \item Il sistema incrementa il numero di copie disponibili del libro.
            \item Il sistema rimuove il libro dalla lista dei prestiti correnti dell'utente.
        \end{enumerate}
    \end{itemize}


    \subsection{Descrizione dei Casi d'Uso sulla Gestione dei Dati}\label{subsec:data-use-cases}

    \begin{figure}[H]
        \centering
        \includegraphics[width=1\linewidth]{UML-data@4x}
        \label{fig:gestioneDati}
    \end{figure}

    \subsubsection{UC-17: Salvataggio Automatico (con Backup)}\label{UC:17}
    \begin{itemize}
        \item \textbf{Attori}: Sistema
        \item \textbf{Precondizioni}: È stata effettuata un'operazione di scrittura.
        \item \textbf{Postcondizioni}: I dati sono salvati su file e il backup è aggiornato.
        \item \textbf{Requisiti Implementati}: \hyperref[FDR:05]{FDR-05}
        \item \textbf{Scenario Principale}:
        \begin{enumerate}
            \item L'utente completa un'operazione di modifica dati.
            \item Il sistema crea automaticamente una copia di backup dei file attuali (se implementato \hyperref[NRR:04]{NRR-04}).
            \item Il sistema sovrascrive i file principali con i nuovi dati.
            \item Il sistema conferma l'avvenuto salvataggio.
        \end{enumerate}
        \item \textbf{Scenari Alternativi}:
        \begin{itemize}
            \item[3a.] Se si è verificato un errore il sistema notifica l'utente che il salvataggio è fallito e mantiene i dati in memoria (se implementato \hyperref[NUR:02]{NUR-02}).
        \end{itemize}
    \end{itemize}

    \subsubsection{UC-18: Caricamento Automatico al Boot}\label{UC:18}
    \begin{itemize}
        \item \textbf{Attori}: Sistema
        \item \textbf{Precondizioni}: L'applicazione viene lanciata.
        \item \textbf{Postcondizioni}: Dati caricati dall'archivio.
        \item \textbf{Requisiti Implementati}: \hyperref[FDR:04]{FDR-04}
        \item \textbf{Scenario Principale}:
        \begin{enumerate}
            \item L'applicazione si avvia.
            \item Il sistema cerca i file di salvataggio.
            \item Il sistema legge e carica i dati di Libri, Utenti e Prestiti.
            \item L'interfaccia grafica viene mostrata con i dati popolati.
        \end{enumerate}
        \item \textbf{Scenari Alternativi}:
        \begin{itemize}
            \item[3a.] Se non esistono file (primo avvio), il sistema inizializza file vuoti.
            \item[3b.] Se i file sono corrotti o illeggibili, il sistema mostra un errore e tenta automaticamente di caricare l'ultima versione di backup funzionante (se implementato \hyperref[NUR:02]{NUR-02}, \hyperref[NRR:05]{NRR-05}).
        \end{itemize}
    \end{itemize}

%===============================================