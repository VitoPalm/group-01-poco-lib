\documentclass[12pt,a4paper]{article}
\usepackage[main=italian,english]{babel}
\usepackage{geometry}
\usepackage{array}
\usepackage{tabularx}
\usepackage{longtable}
\usepackage{fancyhdr}
\usepackage{graphicx}
\usepackage{booktabs}
\usepackage{enumitem}
\usepackage{float}
\usepackage[hidelinks]{hyperref}
\usepackage{xcolor}
\usepackage{tablefootnote}
\usepackage{svg}
\usepackage[htt]{hyphenat}

% Metadata setup
\hypersetup{
    pdftitle={Software Design Document},
    pdfauthor={Marino Francesco, Pepe Daniele, Orsini Giovanni, Palmieri Vito},
    pdfsubject={Software per la gestione di una biblioteca},
    pdfkeywords={SDD, architettura, mvc, classi, sequenza},
    pdfcreator={Gruppo 1},
    pdfproducer={Gruppo 1}
}

% Table settings
\renewcommand{\arraystretch}{1.5}
\setlength{\tabcolsep}{5pt}
\renewcommand{\tabularxcolumn}[1]{m{#1}}

% Page layout
\geometry{left=2.5cm, right=2.5cm, top=3cm, bottom=3cm}
\pagestyle{fancy}
\fancyhf{}
\fancyhead[L]{\textit{Software Design Document}}
\fancyhead[R]{\textit{Gruppo 1}}
\fancyfoot[C]{\thepage}

% Custom colors
\definecolor{sectioncolor}{RGB}{0,102,204}

\title{\textbf{\Large Software Design Document (SDD)}\\
\large Software per la gestione di una biblioteca}
\author{Autori: Marino Francesco, Pepe Daniele, Orsini Giovanni, Palmieri Vito \\ Gruppo 1}

\begin{document}

    \maketitle
    \thispagestyle{empty}

    \newpage
    \tableofcontents
    \newpage

%===============================================
    \section{Architettura del sistema}
%===============================================

    L'architettura del sistema è organizzata secondo il pattern Model-View-Controller (MVC). I moduli principali sono divisi in package che riflettono questa struttura adattata alla complessità del progetto e alle esigenze specifiche dell'applicazione:

    \begin{figure}[H]
        \centering
        \includegraphics[width=0.9\textwidth]{../png/Packages@4x}
        \caption{Diagramma dei Package}\label{fig:figure8}
    \end{figure}

    \begin{table}[H]
        \centering
        \small
        \begin{tabularx}{\textwidth}{|m{3cm}|X|m{3cm}|m{2,5cm}|}
            \hline
            \textbf{Modulo} & \textbf{Responsabilità} & \textbf{Dipendenze} & \textbf{Test} \\
            \hline
            \texttt{poco.company.\newline group01pocolib} & Contiene l'entry point dell'applicazione (\texttt{Launcher}) e coordina l'inizializzazione del sistema. & \texttt{mvc.controller}, \texttt{mvc.model}, JavaFX & - \\
            \hline
            \texttt{mvc.controller} & Gestisce la logica di presentazione e l'interazione utente.
            Include i controller per le proprietà degli elementi (\texttt{BookPropController}, ecc.) e per le tab. Il \texttt{PocoLibController} coordina i controller delle tab e gestisce la navigazione. & \texttt{mvc.model}, JavaFX FXML & - \\
            \hline
            \texttt{mvc.model} & Implementa la logica di business e mantiene lo stato dell'applicazione. Contiene le entità (\texttt{Book}, \texttt{User}, \texttt{Lending}) e le collezioni che le gestiscono. Fornisce metodi per operazioni CRUD e ricerca. & \texttt{db}, Java I/O & \texttt{BookTest}, \texttt{UserTest}, \texttt{LendingTest} (e relativi Set) \\
            \hline
            \texttt{db} & Responsabile della persistenza dei dati su file system (\texttt{DB}), gestione dei backup (\texttt{Backup}) e calcolo hash per l'integrità dei file (\texttt{Hash}). Serializza e deserializza le strutture dati. & \texttt{mvc.model}, Java I/O & \texttt{DBTest}, \texttt{BackupTest}, \texttt{HashTest} \\
            \hline
            \texttt{db.omnisearch} & Implementa la funzionalità di ricerca avanzata tramite indicizzazione trigram (\texttt{Index}) e algoritmi di ricerca fuzzy (\texttt{Search}). Supporta ricerche parziali e tolleranti agli errori di battitura. & \texttt{mvc.model} & \texttt{SearchTest} \\
            \hline
            \texttt{exceptions} & Contiene la gerarchia delle eccezioni personalizzate per la gestione degli errori specifici dell'applicazione (\texttt{BookDataNotValidException}, ecc.). & Nessuna dipendenza interna & - \\
            \hline
        \end{tabularx}\label{tab:table}
    \end{table}

%===============================================
    \section{Modello statico}
%===============================================

    Le classi del sistema sono state organizzate secondo il pattern Model-View-Controller (MVC) per separare la logica di business dalla logica di presentazione e dalla persistenza dei dati.

    \subsection{Diagramma delle classi correlate a Launcher}

    \begin{figure}[H]
        \centering
        \includegraphics[width=0.9\textwidth]{../png/Classes-launcher@4x}
        \caption{Diagramma delle classi correlate a Launcher}\label{fig:figure-launcher}
    \end{figure}

    \subsubsection{Classe \texttt{Launcher}}
    Questa classe contiene il metodo \texttt{main} e il metodo \texttt{start} per avviare l'app JavaFX.
    Contiene inoltre i metodi \texttt{restoreBookSet}, \texttt{restoreUserSet} e \texttt{restoreLendingSet} per caricare i dati.

    \subsection{Diagramma delle classi di MVC}

    \begin{figure}[H]
        \centering
        \includegraphics[width=\textwidth]{../png/Classes-mvc@4x}
        \caption{Class Diagram MVC}\label{fig:figure-mvc}
    \end{figure}

    \subsubsection{Classi di \texttt{mvc.model}}
    Il model contiene le classi che implementano la logica di business. Le principali classi sono:
    \begin{itemize}
        \item \texttt{Book}: È responsabile della rappresentazione di un oggetto libro con attributi come ISBN, titolo, autori, anno di pubblicazione e numero di copie disponibili. Fornisce metodi per accedere e modificare questi attributi.
        \item \texttt{User}: È responsabile della rappresentazione di un utente con attributi come ID, nome, cognome, email e numero di telefono. Fornisce metodi per accedere e modificare questi attributi.
        \item \texttt{Lending}: È responsabile della rappresentazione di un prestito, collegando un \texttt{Book} a un \texttt{User} con attributi come data di prestito e data di restituzione prevista.
        \item \texttt{BookSet}, \texttt{UserSet}, \texttt{LendingSet}: Collezioni responsabili della gestione rispettivamente di libri, utenti e prestiti. Forniscono metodi per aggiungere, rimuovere e cercare elementi all'interno delle collezioni.
    \end{itemize}

    \subsubsection{Classi di \texttt{mvc.controller}}
    Il controller è responsabile del collegamento tra il model (la logica di business) e la view (la logica di presentazione). Le principali classi sono:
    \begin{itemize}
        \item \texttt{BookPropController}, \texttt{UserPropController}, \texttt{LendingPropController}: Gestiscono le operazioni di visualizzazione e modifica delle proprietà. I metodi permettono di gestire azioni come visualizzazione dettagli, salvataggio e annullamento modifiche.
        \item \texttt{BookTabController}, \texttt{UserTabController}, \texttt{LendingTabController}: Gestiscono le operazioni sulle tabelle principali. I metodi permettono di gestire interazioni come reindirizzamento ai dettagli o aggiornamento dati.
        \item \texttt{PocoLibController}: È responsabile della gestione delle interazioni generali dell'applicazione, coordinando le operazioni tra i vari controller specifici di entità.
    \end{itemize}

    \subsubsection{Artifacts di \texttt{mvc.view}}
    La view contiene i file FXML che implementano l'interfaccia grafica dell'app.
    \begin{itemize}
        \item \texttt{BookPropView}, \texttt{UserPropView}, \texttt{LendingPropView}: Finestre di visualizzazione e modifica delle proprietà.
        \item \texttt{BookTabView}, \texttt{UserTabView}, \texttt{LendingTabView}: Tab corrispondenti alla visualizzazione e interazione con le tabelle.
    \end{itemize}

    \paragraph{Esempio delle interazioni di un Model-View-Controller specifico (UserSet)}
    \begin{figure}[H]
        \centering
        \includegraphics[width=0.8\textwidth]{../png/Classes-mvc-specific@4x}
        \caption{Esempio interazioni MVC specifico (UserSet)}\label{fig:figure-mvc-specific}
    \end{figure}

    \paragraph{Esempio delle interazioni di un modello e della sua aggregazione (Book e BookSet)}
    \begin{figure}[H]
        \centering
        \includegraphics[width=0.8\textwidth]{../png/Classes-bookv2@4x}
        \caption{Esempio delle interazioni Book e BookSet}\label{fig:figure-book-bookset}
    \end{figure}

    \subsection{Diagramma delle classi di \texttt{db} e \texttt{db.omnisearch}}\label{subsec:diagramma-delle-classi-di-db-e-db.omnisearch}

    \begin{figure}[H]
        \centering
        \includegraphics[width=\textwidth]{../png/Classes-db@4x}
        \caption{Class Diagram DB}\label{fig:figure2}
    \end{figure}

    \subsubsection{Classi di \texttt{db}}
    Il package \texttt{db} è responsabile della persistenza dei dati e della gestione dei backup.
    \begin{itemize}
        \item \texttt{DB}: Responsabile della creazione, lettura, scrittura e aggiornamento del file di database.
        I metodi principali prevedono la costruzione e aggiornamento del file da una Cache.
        \item \texttt{Backup}: Responsabile della creazione effettiva dei backup del file di database.
        \item \texttt{Hash}: Responsabile del calcolo dell'hash del file di database per garantire l'integrità dei dati.
    \end{itemize}

    \subsubsection{Classi di \texttt{db.omnisearch}}
    Il package \texttt{db.omnisearch} implementa la funzionalità di ricerca avanzata.
    \begin{itemize}
        \item \texttt{Index}: Responsabile dell'indicizzazione trigram per ricerche più efficienti.
        \item \texttt{Search}: Implementa algoritmi di ricerca fuzzy per trovare corrispondenze parziali, permettendo ricerche tolleranti agli errori di battitura.
    \end{itemize}

    \subsection{Diagramma delle classi di \texttt{exceptions}}\label{subsec:diagramma-delle-classi-di-exceptions}

    \begin{figure}[H]
        \centering
        \includegraphics[width=0.8\textwidth]{../png/Classes-exceptions@4x}
        \caption{Class Diagram Exceptions}\label{fig:figure3}
    \end{figure}

    \subsubsection{Classi di \texttt{exceptions}}
    Il package \texttt{exceptions} contiene la gerarchia delle eccezioni personalizzate.
    \begin{itemize}
        \item \texttt{BookDataNotValidException}: Sollevata quando i dati di un libro non sono validi.
        \item \texttt{UserDataNotValidException}: Sollevata quando i dati di un utente non sono validi.
    \end{itemize}

    \begin{quote}
        Tali eccezioni sono state create poiché in questo modo qualsiasi \texttt{model} istanziato conterrà sempre dati validi, nonostante si prevede che il bibliotecario non avrà possibilità di salvare un nuovo \texttt{model} se la validazione dei dati al momento dell'input risulterà errata.
        Il loro inserimento è correlato quindi all'idea che i model sono un componente funzionante autonomamente, poi interfacciato con la GUI.
    \end{quote}

    \subsection{Scelte Progettuali}\label{subsec:scelte-progettuali}

    Il sistema è stato progettato con l'obiettivo di garantire manutenibilità, modularità e facilità di estensione.

    \begin{itemize}
        \item \textbf{Adozione del pattern MVC}: Isola la logica di business, la logica di presentazione e la gestione degli eventi.
        \item \textbf{Progettazione orientata agli oggetti}: Segue principi come incapsulamento, ereditarietà e polimorfismo.
        \item \textbf{Implementazione di un sistema di ricerca avanzata}: L'uso di indici trigram e algoritmi di fuzzy search offre funzionalità di ricerca potenti.
        \item \textbf{Gestione della persistenza}: Classi dedicate (\texttt{DB}, \texttt{Backup}, \texttt{Hash}) isolano la logica di accesso ai dati.
        \item \textbf{Utilizzo di eccezioni personalizzate}: Facilitano la gestione degli errori e migliorano la robustezza.
    \end{itemize}

    \subsection{Principi di buona progettazione}\label{subsec:principi-di-buona-progettazione}
    Il sistema rispetta diversi principi di buona progettazione software:
    \begin{itemize}
        \item \textbf{Single Responsibility Principle (SRP)}: Ogni classe ha una singola responsabilità ben definita (es. \texttt{DB} serializzazione, \texttt{Backup} backup).
        \item \textbf{Don't Repeat Yourself (DRY)}: La logica di validazione è centralizzata nel model; la ricerca fuzzy è implementata una sola volta in \texttt{db.omnisearch}.
        \item \textbf{Separation of Concerns}: Netta separazione tra View, Model e DB.
        \item \textbf{Information Hiding}: I dettagli implementativi delle strutture dati interne sono nascosti dietro interfacce pubbliche.
    \end{itemize}

%===============================================
    \section{Modello dinamico}\label{sec:modello-dinamico}
%===============================================

    In questa sezione vengono presentati i diagrammi UML per i casi d'uso più significativi dell'applicazione.

    \subsection{Diagramma di sequenza: Aggiunta di un nuovo libro}\label{subsec:diagramma-di-sequenza:-aggiunta-di-un-nuovo-libro}
    Nel seguente diagramma viene illustrato il processo di aggiunta di un nuovo libro.
    I dati vengono elaborati dal controller, salvati nel model, e successivamente su file tramite la classe DB, creando anche un backup.
    In caso di errori, viene sollevata un'eccezione gestita dal controller.

    \begin{figure}[H]
        \centering
        \includegraphics[width=\textwidth]{../png/sequence_add_book@4x}
        \caption{Sequence Diagram: Add Book}\label{fig:figure4}
    \end{figure}

    \subsection{Diagramma di sequenza: Prestito di un libro}\label{subsec:diagramma-di-sequenza:-prestito-di-un-libro}
    Il bibliotecario seleziona utente e libro.
    Il controller verifica le condizioni (limite prestiti, disponibilità libro).
    Se soddisfatte, viene creato il prestito, aggiornato lo stato del libro e salvato tutto su file (con backup).
    In caso contrario, viene sollevata un'eccezione.

    \begin{figure}[H]
        \centering
        \includegraphics[width=\textwidth]{../png/sequence_lend_book@4x}
        \caption{Sequence Diagram: Lend Book}\label{fig:figure5}
    \end{figure}

    \subsection{Diagramma di Attività: Avvio dell'Applicazione e Caricamento Dati}\label{subsec:diagramma-di-attivita:-avvio-dell'applicazione-e-caricamento-dati}
    All'avvio, l'app verifica l'esistenza del file di salvataggio.
    Se non esiste, ne crea uno vuoto.
    Se esiste, tenta il caricamento.
    In caso di fallimento del caricamento, tenta il ripristino da backup.
    Se anche il ripristino fallisce, restituisce errore critico e inizializza un nuovo file vuoto.

    \begin{figure}[H]
        \centering
        \includegraphics[width=\textwidth]{../png/activity_startup@4x}
        \caption{Activity Diagram: Startup}\label{fig:figure6}
    \end{figure}

%===============================================
    \section{Design dell’interfaccia utente}\label{sec:design-dellinterfaccia-utente}
%===============================================

    L'interfaccia utente è progettata per essere intuitiva e user-friendly, con un layout chiaro e funzionale.

    \subsection{Schermate principali dell'applicazione}\label{subsec:schermate-principali-dell'applicazione}
    La schermata principale presenta una tabella che elenca i contenuti in base alla sezione selezionata (Prestiti, Libri, Utenti) tramite barra di navigazione.
    Include funzionalità di ricerca testuale (anche parziale/fuzzy) e bottoni per azioni CRUD\@.

    \subsubsection*{Pagina dei Prestiti}
    Permette di visualizzare: Lending ID, Return Date, ISBN, Title, User ID, User.
    Include bottoni per segnare come restituito un libro e per visualizzare/modificare un prestito.

    \begin{figure}[H]
        \centering
        \includegraphics[width=0.9\textwidth]{../../SRS/Resources/Main-page-lending}
        \caption{Main Page: Lending}\label{fig:figure7}
    \end{figure}

    \subsubsection*{Pagina dei Libri}
    Permette di visualizzare: ISBN, Title, Authors, Year, Available, Lent. Include bottoni per aggiungere, visualizzare/modificare e prestare libri.

    \begin{figure}[H]
        \centering
        \includegraphics[width=0.9\textwidth]{../../SRS/Resources/Main-page-books}
        \caption{Main Page: Books}
    \end{figure}

    \subsubsection*{Pagina degli Utenti}
    Permette di visualizzare: ID, Name, Surname, Email, Lent (numero libri in prestito). Include bottoni per aggiungere, visualizzare/modificare e avviare prestiti per l'utente.

    \begin{figure}[H]
        \centering
        \includegraphics[width=0.9\textwidth]{../../SRS/Resources/Main-page-users}
        \caption{Main Page: Users}
    \end{figure}

    \subsection{Finestre di visualizzazione}
    Finestre aperte tramite ``View and Edit''. Mostrano dettagli e offrono collegamenti rapidi agli elementi collegati.

    \begin{itemize}
        \item \textbf{Lending View}: Mostra dettagli prestito. Bottoni per modifica, eliminazione (se restituito), restituzione libro.
        \item \textbf{Book View}: Mostra dettagli libro. Bottoni per modifica, eliminazione (se non in prestito), avvio prestito.
        \item \textbf{User View}: Mostra dettagli utente. Bottoni per modifica, eliminazione (se senza prestiti), avvio prestito.
    \end{itemize}

    \begin{figure}[H]
        \centering
        \includegraphics[width=0.45\textwidth]{../../SRS/Resources/Mockup-view-lending}
        \hfill
        \includegraphics[width=0.45\textwidth]{../../SRS/Resources/Mockup-view-book}
        \caption{Finestre di visualizzazione: Prestito (sx) e Libro (dx)}
    \end{figure}
    \begin{figure}[H]
        \centering
        \includegraphics[width=0.45\textwidth]{../../SRS/Resources/Mockup-view-user}
        \caption{Finestra di visualizzazione: Utente}
    \end{figure}

    \subsection{Finestre di modifica}
    Finestre aperte tramite bottone ``Edit'' nelle schermate di visualizzazione. Permettono di salvare le modifiche ai dati.

    \begin{figure}[H]
        \centering
        \includegraphics[width=0.3\textwidth]{../../SRS/Resources/Mockup-edit-lending}
        \hfill
        \includegraphics[width=0.3\textwidth]{../../SRS/Resources/Mockup-edit-book}
        \hfill
        \includegraphics[width=0.3\textwidth]{../../SRS/Resources/Mockup-edit-user}
        \caption{Finestre di modifica: Prestito, Libro, Utente}
    \end{figure}

\end{document}